% Vorstellung des SZ in Lehrveranstaltungen
\section*{Schreibzentrum}
\label{sec:schreibzentrum}
\subsection*{Wer oder was ist das Schreibzentrum der TU Dresden?}
\label{sub:wer_oder_was_ist_das_schreibzentrum_der_tu_dresden_}
Das Schreibzentrum ist ein Kooperationsprojekt an der TU Dresden für Studierende und Lehrende
vom Zentrum für Weiterbildung und dem Career Service. Wir bieten Unterstützung, Methoden und
Ideen zum Thema \enquote{wissenschaftliches Schreiben}.

\subsection*{Was ist Schreibberatung?}
Studierende aller Fachrichtungen können mit ihren individuellen Schreibprojekten aller Art (Beleg,
Seminararbeit, Abschlussarbeit, etc.) entweder in die offene Schreibsprechstunde am SCS
ServicePoint der SLUB kommen oder einen individuellen Termin per E-Mail vereinbaren. Dabei spielt
es keine Rolle, wie weit die Arbeit bereits ist, ob man also noch ganz am Anfang steht oder kurz vor
der Abgabe. Auch muss kein konkretes Problem vorliegen, sondern intuitive Anliegen zur Arbeit
können ebenfalls geschildert werden.

Die Schreibberatung unterstützt bei Fragen zum Schreibprozess -- von der Themenfindung, über die
Gliederung bis hin zur Abgabe der fertigen Arbeit. Ausgebildete studentische Schreibtutorinnen und
Schreibtutoren unterstützen euch mit vielfältigen Schreibmethoden und Techniken.

Im Gespräch finden wir gemeinsam eine Antwort auf eure Fragen. Das Angebot ist selbstverständlich
kostenlos.

Die Schreibberatung kann keine inhaltlichen Tipps oder Hilfestellungen geben und liest keine Texte
Korrektur. Ihr könnt allerdings exemplarisch Textfeedback auf Textauszüge bekommen.

\subsection*{Wie könnt ihr uns kontaktieren?}
\label{sub:wie_konnt_ihr_uns_kontaktieren_}
\begin{itemize}
  \item \href{mailto:Schreibzentrum@mailbox.tu-dresden.de}{\nolinkurl{Schreibzentrum@mailbox.tu-dresden.de}}
    \item \url{https://facebook.de/SchreibzentrumTUD}
    \item Offene Schreibsprechstunde am SCS ServicePoint in der SLUB: Dienstags, 11.15-12.15 Uhr
    
\end{itemize}

