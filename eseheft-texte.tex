%! TEX program = lualatex

\documentclass{scrartcl}
% \usepackage[T1]{fontenc}
% \usepackage[utf8]{inputenc}
\usepackage[ngerman]{babel}
\usepackage{fontspec}
\setmainfont{Linux Libertine O}
\usepackage{hyperref}
\usepackage{microtype}
\usepackage{enumitem}
\usepackage[a5paper, margin=1cm, right=3cm]{geometry}
\usepackage{amssymb}
\usepackage{scrextend} % for footnotes, see https://tex.stackexchange.com/questions/35043/reference-different-places-to-the-same-footnote
\setlength{\emergencystretch}{1em}
\usepackage{graphicx}
\usepackage{pdfpages}
\usepackage{textcomp} % für \textmu
\setlist{noitemsep} % Listen werden kompakter

\begin{document}
\newcommand{\keyword}{\emph}
\newcommand{\todo}[1]{\marginpar{TODO\\#1}}
% TODO:  <06-09-17, Bela> Deckblatt: μFSR-Bild, dafür nach Möglichkeit ein neues machen (Bene, Luisa, Felix sind weg, Assoziierte können mit rauf) %
\newpage
{ \newgeometry{margin=2cm}
  \thispagestyle{empty} % no page number
  % Titlepage, all formatting should be restricted to this page.
  \begin{flushright}
    \includegraphics[width=0.3\textwidth]{./fsrlogos/bw_1024.png}
  \end{flushright}
  \begin{center}
    \begin{huge}
      Erstsemesterinfos Mathematik
    \end{huge}\\\bigskip
    \begin{Large}
      vom FSR Mathematik
    \end{Large}\\\bigskip
%    \includegraphics[width=\textwidth]{./fsrfoto.jpg}
  \end{center}
  TODO: Deckblatt: μFSR-Bild, dafür nach Möglichkeit ein neues machen (Bene, Luisa, Felix sind weg, Assoziierte können mit rauf) 
  \vfill
  \begin{small}
    Dieses Heft kann auch als pdf\footnote{\url{TODO: link}} heruntergeladen werden.
  \end{small}
  \newpage
}
restoregeometry % original margins
\pagenumbering{arabic} % resets numbering to 1
\pagestyle{plain}
\section*{\texorpdfstring{$\mu$}{my}FSR -- Fachschaftsrat Mathematik}
Alle Studierenden der Mathematik an der TU Dresden sind die \keyword{Fachschaft} Mathematik.
Sie werden vertreten und unterstützt vom Fachschaftsrat Mathematik, kurz μFSR.

Der μFSR ist erreichbar per e-Mail\footnote{\href{mailto:kontakt@myfsr.de}{\nolinkurl{kontakt@myfsr.de}}},
Internetseite\footnote{\url{https://myfsr.de}} und vor allem persönlich im Raum
\href{https://navigator.tu-dresden.de/etplan/wil/-1/raum/2193-1.0170}{WIL B 23}.

\subsection*{Unterstützung beim Studium}
\label{sub:unterstutzung_beim_studium}
\begin{description}
  \item[Studienberatung] Wir sind Studierende wie du, aber teilweise schon etwas länger hier.
    Daher können wir dir bei vielen Fragen zu Studienablauf, Prüfungsorganisation und -ordnung und Ansprechpartnern in der Uni weiterhelfen.
    Wir freuen uns dich zu unterstützen.
    Dazu kann auch gehören bei Konflikten zwischen Studierenden und Dozenten zu vermitteln.
  \item[Veranstaltungen] Wir bieten Veranstaltungen an wie beispielsweise
    \begin{itemize}
      \item Spiele-, Skatabende
      \item Weihnachtsfeier
      \item Erstsemestereinführung inkl.\ Ersti-Fahrt
      \item Winterball
      \item Sportturniere, z.B.\ Volleyball, Fußball
      \item Wir unterstützen gerne deine Initiative organisatorisch und/oder finanziell.
    \end{itemize}
    Achte auf die Plakate an den Willersbaueingängen und an der Pinnwand im EG des C-Flügels und
    abonniere den νsletter des μFSR\footnote{\url{https://myfsr.de/minitrue}}.
  \item[Klausuren] Klausuren von vorherigen Jahren sind hilfreich bei der Prüfungsvorbereitung.
    Wir sammeln Mitschriften und stellen sie euch zur Verfügung.
    Schreibe einfach eine e-Mail an \href{mailto:klausur@myfsr.de}{\nolinkurl{klausur@myfsr.de}}!
  \item[$\nu$sletter] Mit dem Newsletter des μFSR informieren wir euch zum einen über
    Veranstaltungen des μFSR und zum anderen über die interessantesten Angebote der Uni,
    die uns erreichen oder die wir finden.
\end{description}

\subsection*{Hochschulpolitische Vertretung}
\label{sub:hochschulpolitische_vertretung}
Der μFSR setzt sich bei Entscheidungen in der Uni, in erster Linie innerhalb der Fakultät Mathematik,
für die Interessen der Studierenden ein.
Dafür hat er Vertreter in verschiedenen Gremien\footnote{\url{www.myfsr.de/dokuwiki/doku.php?id=hopo:gremien}}.
Wenn man sich in diesem Bereich \keyword{engagiert} bekommt man vieles von dem mit, was hinter den Kulissen
des Unialltags passiert und kann einiges bewegen.
Auch ohne politische oder Unierfahrung kannst du dich einbringen und spannende Aufgaben übernehmen!

\subsection*{Mitmachen}
\label{sub:mitmachen}
Wir freuen uns, wenn du dich einbringen möchtest! Jede/r kann aktiv werden.
Dadurch, dass jedes Engagement freiwillig ist, kann man sich auf die eigenen Interessen konzentrieren
und mit Gleichgesinnten zusammenarbeiten. Mögliche Aufgaben sind
\begin{itemize}
  \item Das, was du bewegen möchtest!
  \item Die Veranstaltung, die du haben möchtest.
  \item Finanzen, Sprecher (Voraussetzung: gewähltes Mitglied)
  \item Sitzungsleitung, Protokollant (idealerweise kein gewähltes Mitglied)
  \item Gremienmitgliedschaft (manche direkt gewählt, andere vom FSR entsandt)
  \item \dots
\end{itemize}
Bei den meisten Themen ist es unerheblich, ob man als gewähltes oder nicht-gewähltes („assoziiertes“) Mitglied dabei ist.

\subsection*{Fachschaft bei Lehramt}
\label{sub:fachschaft_bei_lehramt}
Wenn du einen Lehramtsstudiengang studierst, kommen mehrere Fachschaften für dich in Frage:
\begin{itemize}
  \item Mathematik
  \item dein zweites Fach
  \item Allgemeinbildende Schulen (ABS) für das Lehramt an Grund-, Mittelschule und Gymnasium bzw.
  \item Berufspädagogik (BP) für das Lehramt an berufsildenden Schulen 
\end{itemize}
Zu welcher Fachschaft du gehörst, hast du bei der Immatrikulation entschieden,
du kannst es aber jederzeit ändern.
Dazu stellt der zuständige studentische Wahlausschuss das Formular
\keyword{Antrag auf Wechsel der Fachschaft}\footnote{\url{www.stura.tu-dresden.de/wahlen}} online.
  Die Fachschaftszugehörigkeit ist in erster Linie bei den Hochschulwahlen relevant,
  denn du kannst nur bei deiner eingetragenen Fachschaft wählen und gewählt werden.
% Wenn du einen Lehramtsstudiengang studierst, kommen mehrere Fachschaften für dich in Frage: Mathematik, dein zweites Fach und Allgemeinbildende Schulen.
% Zu welcher Fachschaft du gehörst, hast du bei der Immatrikulation entschieden, du kannst es aber jederzeit ändern.
% In erster Linie ist es bei den Hochschulwahlen relevant, denn dort kannst du nur bei deiner offiziellen
% eigenen Fachschaft wählen und gewählt werden.

\subsection*{Hochschulgruppen}
\label{sub:hochschulgruppen}
Neben den hochschulpolitischen Vertretungen (FSR, Studentenrat (StuRa)) gibt es zahlreiche Hochschulgruppen,
die sich für verschiedenes einsetzen. Manche stellen sich in diesem Heft vor,
alle findet man auf der Internetseite des StuRa\footnote{\url{www.stura.tu-dresden.de/hochschulgruppen}}.

\section*{Angebote der Universität: Checkliste}
\label{sec:angebote_der_universitat}
Die TU Dresden bietet eine Menge Services an, die sehr praktisch für das Studium und auch drum herum sein können.
Man muss sie nur kennen. Hier ist eine Liste von Angeboten, die du für dich einrichten solltest.
Eine deutlich umfangreichere Informationssammlung zum Studienstart
existiert online\footnote{\url{www.tu-dresden.de/studienstart}} von der Uni.
\newcommand{\clitem}[1]{\item[$\square$ #1]}
% \newcommand{footnoteref}[1]{\footnotemark[\ref{#1}]}
\begin{description}
  \clitem{ZIH-Login aktivieren} Das Zentrum für Informationsdienste und Hochleistungsrechnen (\keyword{ZIH})
    bietet alle IT-Dienste für die TUD an. Man braucht daher einen \keyword{ZIH-Login}, den bei der
    Immatrikulation bekommen hat, aber noch aktiviert werden muss.
    Dieser wird häufig als \keyword{s-Nummer} bezeichnet, da es eine Nummer mit vorangestelltem „s“ ist.
    Informationen dazu und zu allen (folgenden) IT-Diensten finden sich ausführlich auf den
    Erstsemesterseiten des ZIH \footnote{\url{https://tu-dresden.de/zih/dienste/service-desk/ese}\label{zihurl}}.
  \clitem{Uni-e-Mail-Adresse} Du hast eine e-Mail-Adresse\footref{zihurl} der Form \url{vorname.nachname@mailbox.tu-dresden.de}.
    Zum Einen ist es praktisch für Kommunikation in Uniangelegenheiten dafür eine e-Mail-Adresse zu haben.
    % Beispielsweise bittet das zentralisierte Lehrerprüfungsamt darum diese Adresse für die Kommunikation zu nutzen.
    Zum Anderen bist du von der Uni verpflichtet, diese regelmäßig zu lesen.
    Beispielsweise kommt darüber rechtzeitig die Erinnerung, sich für das kommende Semester zurückzumelden.
    Du kannst die eingehenden e-Mails auf deine private e-Mail-Adresse umleiten oder sie wie andere Adressen
    online\footnote{\url{https://msx.tu-dresden.de}}\footnote{Login mit s-Nummer\label{snr}} oder über einen e-Mail-Client wie Thunderbird abrufen (unsere Empfehlung).
  \clitem{eduroam} Auf dem gesamten Unigelände und in zahlreichen Unis auf der ganzen Welt steht dir
    das WLAN \keyword{eduroam}\footref{zihurl} mit Internetzugang zur Verfügung.
  \clitem{$\nu$sletter} Melde dich doch für den Newsletter des μFSR an\footnote{https://myfsr.de/minitrue}.
    Wir spammen dich nicht voll, sondern
    filtern aus den zahlreichen Angeboten für Studierende die für dich (hoffentlich) spannensten heraus.
    % Das ist einer der sinnvollen Verwendungszwecke für die Uni-e-Mail-Adresse.
  \clitem{SLUB-Karte} Die Sächsische Landes- und Universitätsbibliothek (SLUB) hat ihre eigene Benutzerkarte,
    die man sich ausstellen lässt. Du füllst erst das Anmeldeformular aus und holst dann die SLUB-Karte ab.
    Dabei an Perso und Immatrikulationsbescheinigung denken.
    Infos auf \url{www.slub-dresden.de/service/nutzer-der-slub-werden}.
  \clitem{Mensa-Karte} Du kannst theoretisch in der Mensa mit Bargeld bezahlen,
    aber das ist unpraktisch und hält den eh vollen Betrieb auf und
    du müsstest dich jedes Mal als Student ausweisen.
    Deshalb gibt es die Mensa-Karte,
    die du in jeder Mensa für 5€ Kaution bekommst.
    Um Studentenpreise zu zahlen musst du die EMeal-Bescheinigung von deinem Immatrikulationsbogen
    einmal am Anfang jedes Semesters vorzeigen.
    % Studentenausweis oder Immatrikulationsbescheinigung reichen nicht.
  \clitem{OFP} Die Orientierungsplattform Forschung und Praxis (OFP) bietet regelmäßig spannende Veranstaltungen
    speziell für uns Mathematikstudierende an.
    Diese kann man leicht übersehen und verpassen,
    daher ist es eine gute Idee sich für den Newsletter anzumelden.
    Das geht indem man sich bei OPAL\footref{snr} anmeldet und zum passenden auf den
    OFP-Seite\footnote{\url{www.tu-dresden.de/tu-dresden/profil/exzellenz/zukunftskonzept/tud-structures/zill/orientierungsplattform-forschung-praxis/mathematik}}
    verlinkten Kurs\footnote{\url{https://bildungsportal.sachsen.de/opal/auth/RepositoryEntry/12520161286/CourseNode/94473775961368}} geht.
    
    Ja, OPAL ist etwas umständlich zu bedienen.
  \clitem{Sportkurse} Das Universitätssportzentrum (\keyword{USZ})\footnote{\url{www.tu-dresden.de/usz/zentrale-infos/hinweise-zur-buchung}} bietet ein riesiges Angebot\footnote{\url{www.usz.tu-dresden.de}} an
    Sportkursen an, die für Studierende auch sehr günstig sind.
    Sehr viele dieser Kurse sind sehr beliebt, weshalb man sehr schnell bei der Anmeldung sein muss.
    Sehr schnell bedeutet, dass innerhalb von unter 10 Sekunden alle Plätze vergeben sein können.
    \todo{ wie heißt die Schnellanmeldefunktion? Die ist als Kurs registriert. } % TODO:  <30-08-17, Bela> wie heißt die Schnellanmeldefunktion? Die ist als Kurs registriert. %
    Der Anmeldebeginn steht bei dem jeweiligen Kurs dabei. Die ersten beginnen am 10.10.2017 um 17 Uhr.
  \clitem{Kulturangebote} Es gibt ca.\ 30 künstlerische Gruppen\footnote{\url{https://www.studentenwerk-dresden.de/kultur/gruppen.html}} bei denen du dich ausleben kannst.
\end{description}
Die folgende Punkte muss man nicht unbedingt direkt am Anfang des Semesters beachten, sollte man
sich aber bald anschauen:
\begin{description}
  \item[Campus-Navigator] Oft ist es nicht einfach einen bestimmten Raum zu finden,
    insbesondere in selten besuchten Gebäuden für Nebenfächer.
    Daher gibt es den sehr praktischen Campus-Navigator.\footnote{\url{https://navigator.tu-dresden.de/}}
  \item[Nebenfach] Zum Mathematikstudium gehört ein Nebenfach.
    Für dieses gibt es mehrere Vorschläge, die in der Anlage 2 der Prüfungsordnung
    \footnote{\url{www.verw.tu-dresden.de/AmtBek/PDF-Dateien/2016-03/01soBA26.02.2016.pdf\#page=64}}
    nachzulesen sind.
    Du kannst aber auch andere schon belegte Module als Nebenfach anrechnen lassen, solange sie in der Summe
    die gleiche Anzahl an Leistungspunkten haben.
    Bevor du ein nicht vorgeschlagenes Modul belegst wende dich an das Prüfungsamt (Frau Schreiter)\footnote{\url{www.tu-dresden.de/mn/math/studium/pruefungsaemter}\label{pruefungsamt}}
    und kläre ob es angerechnet werden kann und auf welche Art
    du dich zur Prüfung anmelden kannst.

    Für dein Nebenfach entscheidest du dich mit deiner ersten
    Prüfungsanmeldung.
  \item[ownCloud] Unter anderem bietet das ZIH dir 2GB Online-Speicher\footref{zihurl} an.
    Dieser läuft unter dem Name CloudStore\footnote{\url{https://cloudstore.zih.tu-dresden.de}\footref{snr}} (beziehungsweise dem Namen der verwendeten Software ownCloud).
    Dieser ist gegenüber kommerziellen Anbietern zu bevorzugen,
    da hierbei die Daten auf Uni-Servern bleiben. Man kann einzelne Ordner zum Austausch freigeben.
  \item[OPAL] OPAL\footnote{\url{https://bildungsportal.sachsen.de/opal}\footref{snr}} wird zum Einschreiben in Übungen und Hochladen von Übungszetteln genutzt. Theoretisch geht noch viel mehr
    wie Online-Tests und -Übungen.
  \item[SELMA] SELMA\footnote{\url{https://selma.tu-dresden.de}\footref{snr}} heißt Selbstmanagement und ist ein neues Portal für das Organisatorische
    im Studium. Mathematik-Studierende brauchen es zum Herunterladen der Immatrikulationsbescheinigung
    und den Informationen zur Rückmeldung zum nächsten Semester und zum Melden einer neuen Adresse bei Umzug.
  \item[HISQIS] HISQIS\footnote{\url{https://qis.dez.tu-dresden.de}\footref{snr}} ist das Portal zum Anmelden zu Prüfungen. Es soll irgendwann durch SELMA abgelöst werden.
  \item[Semesterticket] Mit dem Semesterticket kannst du den ÖPNV nutzen. Welche Busse, Bahnen und Fahrräder dazugehören und wann du wo ein Fahrrad mitnehmen kannst,
    steht ausführlich auf der StuRa-Seite.\footnote{\url{www.stura.tu-dresden.de/semesterticket}}.
  \item[Lehrveranstaltungskatalog] Der Lehrveranstaltungskatalog\footnote{\url{www.math.tu-dresden.de/math/lvk/lv-abfrage-wi17.htm}} enthält Orte und Zeiten aller angebotenen
    Vorlesungen, Übungen, Seminare, etc. Jede Fakultät hat einen eigenen.
  \item[Studienfachberatung] Die Studienfachberatung berät zu
    inhaltlichen (Welche Module sind für mich sinnvoll, etc.?)
    und organisatorischen Fragen.
    In einigen solcher Fälle kann der μFSR auch weiterhelfen.

    Für den Bachelor Mathematik bietet diese Dr.\ Hans-Peter Scheffler%
    \footnote{\url{www.tu-dresden.de/mn/math/studium/beratung/bachelor_mathematik}}, für das Lehramt Mathematik Dr.\ Karin Weigel\footnote{\url{www.tu-dresden.de/mn/math/studium/beratung/bachelor_mathematik}}.
  \item[Zentrale Studienberatung] Die zentrale Studienberatung\footnote{\url{www.tu-dresden.de/studium/im-studium/beratung-und-service/zentrale-studienberatung}} ist der erste Ansprechpartner wenn über einen Studiengangswechsel nachgedacht wird. Daneben bietet sie aber auch weitere Angebote,
    die man sich anschauen sollte.
    Nichtsdestotrotz haben die meisten innerhalb des Mathematikstudiums keinen Kontakt dahin.
  \item[Career-Service] Der Career-Service\footnote{\url{www.tu-dresden.de/studium/im-studium/beratung-und-service/career-service}} bietet in erster Linie Unterstützung beim Berufseinstieg.
    Aber er bietet auch zahlreiche Workshops\footnote{\url{www.tu-dresden.de/karriere/berufseinstieg/ressourcen/dateien/flyer/2017/semesterprogramm-des-career-service-winter-2017-18\#page=2}} an, die bereits im Studium
    sehr hilfreich sein können wie bspw.\ zu Prüfungsbewältigung und
    wissenschaftlichem Arbeiten und Schreiben.
  \item[Schreibzentrum] Das Schreibzentrum bietet Unterstützung beim Schreiben von wissenschaftlichen Arbeiten, siehe auch deren Seite.
  \item[Studiendokumente] Um nicht von bisher unbekannten Regelungen
    überrascht zu werden, empfiehlt es sich, die Studiendokumente
    \footnote{Bachelor: \url{www.tu-dresden.de/mn/studium/studiendokumente-formulare/mathematik-bachelor}\\
    Lehramt: \url{www.tu-dresden.de/zlsb/lehramtsstudium/studiendokumente}}
    (Studien- und Prüfungsordnung) zur Verfügung stehen zu haben und
    gelesen zu haben.
  \item[Prüfungsamt] Für alle Prüfungen für Mathematik Bachelor Studierende ist
    Frau Schreiter\footref{pruefungsamt} zuständig.
    Die übliche Anmeldung zu (schriftlichen) Prüfungen erfolgt über HISQIS,
    bei Fragen und wenn man sich nicht rechtzeitig angemeldet hat, hilft Frau Schreiter weiter.
    Sie ist äußerst hilfsbereit und macht vieles möglich.
    Daher möchten wir ihr möglichst wenig Ärger machen, das heißt melde dich rechtzeitig im HISQIS an,
    aber zögere nicht,
    dich an Frau Schreiter\footnote{\href{mailto:karola.schreiter@tu-dresden.de}
    {\nolinkurl{karola.schreiter@tu-dresden.de}}} (WIL A 303, Mo bis Do 12:30---14:00) zu wenden.
    Da die Prüfungsämter des Bereichs MatNat neu zusammengezogen sind, ändert sich möglicherweise ihre
    Organisation. Wir sind auch noch gespannt, was das genau bedeutet.

    Wenn du Mathematik auf Lehramt studierst, ist das zentralisierte Lehrerprüfungsamt zuständig.
    \footnote{\url{www.tu-dresden.de/zlsb/die-einrichtung/studienbuero-lehramt/zentralisiertes-lehrerpruefungsamt-zlpa}}
\end{description}
Wenn die hier gegebenen Informationen mehr auf den Bachelor
Mathematik zugeschnitten sind als auf Lehramt Mathematik,
dann tut uns das Leid. Es liegt daran, dass
die meisten μFSR-ler kein Lehramt studieren und sich deshalb
nicht so gut auskennen. Du kannst das ändern, indem du den
μFSR als gewähltes oder assoziiertes Mitglied unterstützt!

\section*{Termine}
\subsection*{ESE-Woche}
\begin{description}
  \item[ESE] \todo{ von Markus endgültige Liste bekommen } % TODO:  <05-09-17, Bela> von Markus endgültige Liste bekommen %
  \item[KrETa] \todo{ spannendes aus KrETa raussuchen } % TODO:  <05-09-17, Bela> spannendes aus KrETa raussuchen %
  \item[Alternative Stadtrundgänge] Am Mi, 11., Fr., 13., Mo., 16. und Di., 17. Oktober, jeweils 17 Uhr
    bietet \keyword{Konsum Global} konsumkritische Stadtrundgänge durch die Neustadt an. Dauer: 2 Stunden,
    Treffpunkt Jorge-Gomondai-Platz beim Albertplatz. Anmeldung\footnote{\href{mailto:konsumglobal_dd@gmx.de}{\nolinkurl{konsumglobal_dd@gmx.de}}} erwünscht, aber nicht zwingend.
\end{description}

\subsection*{Wintersemester}

\begin{itemize}
  \item Hochschulwahlen: voraussichtlich 28.11.2018---30.11.2018. Gewählt wird:
    \todo{ bitte korrigieren, falls etwas fehlt } % TODO:  <05-09-17, Bela> bitte korrigieren, falls etwas fehlt %
    \begin{itemize}
      \item FSR (11 Mitglieder)
      \item Fakultätsrat Mathematik (4 Studierende)
      \item Bereichsrat Mathematik und Naturwissenschaften (4 Studierende)
      \item Senat, erweiterter Senat
    \end{itemize}
    Wenn du dich in eines diesen Gremien wählen lassen möchtest, melde dich bei uns!
  \item Weihnachtsfeier X-Math-Party
    \todo{ Termin für Weihnachtsfeier } % TODO:  <06-09-17, Bela> Termin für Weihnachtsfeier %
  \item Prüfungsanmeldung: Januar 
    \todo{ Zeitraum der Anmeldung zu Prüfungen } % TODO:  <05-09-17, Bela> Zeitraum der Anmeldung zu Prüfungen %
  \item Vorlesungsfreie Zeit: 21.12.2017 bis 3.1.2018 und ab 4.2.2018
  \item Kernprüfungszeit: 5.2.2018 bis 3.3.2018
  
\end{itemize}

\newpage
% \includegraphics[width=0.4\textwidth]{../stav/logo_STAV.jpg} 
\includegraphics[width=0.55\textwidth]{../stav/Gruppenbild_1_STAV.jpg}
\hfill
\includegraphics[width=0.4\textwidth]{../stav/stav_logo_25jahre.jpg}

Hallo,

ich bin dein neuer Nebenjob! 

Du findest mich bei der STAV in der Stura-Baracke gleich neben der HSZ-Wiese,
Montag bis Freitag von 9--15 Uhr kannst du dort im Büro vorbeikommen und
dich für einen Nebenjob vermitteln lassen.
Im Angebot sind die verschiedensten Jobs -- von Umzugshelfer über Nachhilfelehrer
bis zum Werkstudentenjob oder dem Helferjob auf einem Festival hat die STAV fast alles im Angebot.
Eine Übersicht aller Jobangebote und weitere Infos findest du unter:
\url{www.stav-dresden.de}.
Dort wird auch Schritt für Schritt erklärt wie du zu deinem neuen Nebenjob kommst,
es ist also ganz einfach.
Die STAVies freuen sich dir einen passenden Job zu vermitteln – ob kurz-, mittel- oder
langfristig – die Auswahl ist groß und bunt gemischt.
Die meisten Jobangebote liegen mittlerweile auch schon bei 10€/ Stunde oder mehr.

Bis bald!

Deine STAV

PS: Zu Weihnachten vermitteln wir auch Weihnachtsmänner und -engel.
Wenn du also gerne an Weihnachten ein paar Kinder glücklich machen willst,
schau bei unserer Weihnachtsvermittlung vorbei: www.weihnachtsmann-dresden.de







\newpage
\includegraphics[width=\textwidth]{../arbeiterkind/arbeiterkindlogo.jpg}
\subsection*{Dafür engagieren wir uns}
Von 100 Akademiker-Kindern, die die Sekundarstufe 2 erreichen,
beginnen 77 Kinder ein Studium.
Dagegen beginnen von 100 Nicht-Akademiker-Kindern nur 23 Kinder ein Studium.
Die Ursache dafür sind unter anderem Vorurteile, Informationsdefizite und fehlende Vorbilder in der Familie.  
Das möchten wir ändern!

\subsection*{Deshalb bieten wir an}
An Infoständen und in Vorträgen informieren wir über das Studium.
An unserem monatlichen Stammtisch unterstützen wir bei Fragen und Schwierigkeiten rund um das Studium.
Darüber hinaus organisieren wir regelmäßig den
„Tag der Studienfinanzierung“,
bei dem man sich über die verschiedenen Möglichkeiten der Studienfinanzierung beraten lassen kann.

\subsection*{Das sind wir}
Als Studenten und Absolventen, die (fast) alle selbst Arbeiterkinder sind,
geben wir unsere eigenen Erfahrungen im Umgang mit den diversen Herausforderungen eines Studiums weiter. 

\subsection*{Möchtest du uns unterstützen?}
Dann komm zu unserem Stammtisch: jeden 1.\ Mittwoch im Monat um 20 Uhr im Studentenclub Wu5 e.V.,
August-Bebel-Straße 12, 01219 Dresden.\\
Oder kontaktiere uns per E-Mail: \href{mailto:dresden@arbeiterkind.de}{\nolinkurl{dresden@arbeiterkind.de}}

Hier findest du weitere Infos:
\url{www.facebook.com/arbeiterkinddresden},
\url{www.arbeiterkind.de}


\newpage
\includepdf{../psychoberatung/Plakate-2-pdfjam.pdf}
\end{document}
