%! TEX program = lualatex

% TODO:  <31-08-17, Bela> \url beibringen, dass das Internetlinks sind %
% Deckblatt: μFSR-Bild, dafür nach Möglichkeit ein neues machen (Bene, Luisa, Felix sind weg, Assoziierte können mit rauf)
\documentclass{scrartcl}
% \usepackage[T1]{fontenc}
% \usepackage[utf8]{inputenc}
\usepackage[ngerman]{babel}
\usepackage{fontspec}
\setmainfont{Linux Libertine O}
\usepackage{hyperref}
\usepackage{microtype}
\usepackage{enumitem}
\usepackage[a5paper, margin=1cm]{geometry}
\usepackage{amssymb}
\usepackage{scrextend} % for footnotes, see https://tex.stackexchange.com/questions/35043/reference-different-places-to-the-same-footnote
\setlength{\emergencystretch}{1em}

\begin{document}
\newcommand{\keyword}{\emph}
\section{\texorpdfstring{$\mu$}{my}FSR -- Fachschaftsrat Mathematik}
Alle Studierenden, die an der TU Dresden Mathematik studieren, sind die \keyword{Fachschaft} Mathematik.
Sie werden vertreten und unterstützt vom Fachschaftsrat Mathematik, kurz μFSR.

\subsection{Unterstützung beim Studium}
\label{sub:unterstutzung_beim_studium}
\begin{description}
  \item[Studienberatung] Wir sind Studierende wie du, aber teilweise schon etwas länger hier.
    Daher können wir dir bei vielen Fragen zum Studienablauf, Prüfungsorganisation und -ordnung, Ansprechpartnern in der Uni weiterhelfen.
    Frage und wir freuen uns dich zu unterstützen.
    Dazu kann auch gehören bei Konflikten zwischen Studierenden und Dozenten zu vermitteln.
  \item[Veranstaltungen] Wir bieten Veranstaltungen an wie beispielsweise
    \begin{itemize}
      \item Spiele-, Skatabende
      \item Weihnachtsfeier
      \item Erstsemestereinführung inkl.\ Ersti-Fahrt
      \item Winterball
      \item Sportturniere, z.B.\ Volleyball, Fußball
      \item Wir unterstützen gerne deine Initiative organisatorisch und finanziell.
    \end{itemize}
    Achte einfach auf die Plakate an den Willersbaueingängen und an der Pinnwand im EG des C-Flügels und
    abonniere den νsletter des μFSR\footnote{\url{myfsr.de/minitrue/auth/page/email/register\#}}.
    % TODO:  <30-08-17, Bela> Kurzlink νsletter %
  \item[Klausuren] Klausuren von vorherigen Jahren sind hilfreich bei der Prüfungsvorbereitung.
    Wir sammeln Mitschriften und stellen sie euch zur Verfügung.
    Schreibe einfach eine e-Mail an \url{klausur@myfsr.de}!
  \item[νsletter] Mit dem Newsletter des μFSR informieren wir euch zum einen über
    Veranstaltungen des μFSR und zum anderen über die interessantesten Angebote der Uni,
    die uns erreichen oder die wir finden.
\end{description}

\subsection{Hochschulpolitische Vertretung}
\label{sub:hochschulpolitische_vertretung}
Der μFSR setzt sich bei Entscheidungen in der Uni, in erster Linie innerhalb der Fakultät Mathematik,
für die Interessen der Studenten ein.
Dafür hat er Vertreter in verschiedenen Gremien\footnote{\url{myfsr.de/dokuwiki/doku.php?id=hopo:gremien}}.
Wenn man sich in diesem Bereich \keyword{engagiert} bekommt man vieles von dem mit, was hinter den Kulissen
des Unialltags passiert und kann einiges bewegen.
Auch ohne politische oder Unierfahrung kannst du dich einbringen und spannende Aufgaben übernehmen!

\subsection{Mitmachen}
\label{sub:mitmachen}
Wir freuen uns, wenn du dich einbringen möchtest! Jede(r) kann aktiv werden.
Dadurch, dass jedes Engagement freiwillig ist, kann man sich auf die eigenen Interessen konzentrieren
und mit Gleichgesinnten zusammenarbeiten. Mögliche Aufgaben sind
\begin{itemize}
  \item Das, was du bewegen möchtest!
  \item Die Veranstaltung, die du haben möchtest.
  \item Finanzen, Sprecher (Voraussetzung: gewähltes Mitglied)
  \item Sitzungsleitung, Protokollant (idealerweise kein gewähltes Mitglied)
  \item Gremienmitgliedschaft (manche direkt gewählt, andere vom FSR entsandt)
  \item \dots
\end{itemize}
Bei den meisten Themen ist es unerheblich, ob man als gewähltes oder nicht-gewähltes („assoziiertes“) Mitglied dabei ist.

\subsection{Fachschaft bei Lehramt}
\label{sub:fachschaft_bei_lehramt}
Wenn du einen Lehramtsstudiengang studierst, kommen mehrere Fachschaften für dich in Frage:
\begin{itemize}
  \item Mathematik
  \item dein zweites Fach
  \item Allgemeinbildende Schulen (ABS) für das Lehramt an Grund-, Mittelschule und Gymnasium bzw.
  \item Berufspädagogik (BP) für das Lehramt an berufsildenden Schulen 
\end{itemize}
Zu welcher Fachschaft du gehörst, hast du bei der Immatrikulation entschieden,
du kannst es aber jederzeit ändern.
Dazu stellt der zuständige studentische Wahlausschuss das Formular
\keyword{Antrag auf Wechsel der Fachschaft}\footnote{\url{www.stura.tu-dresden.de/wahlen}} online.
  Die Fachschaftszugehörigkeit ist in erster Linie bei den Hochschulwahlen relevant,
  denn du kannst nur bei deiner eingetragenen Fachschaft wählen und gewählt werden.
% Wenn du einen Lehramtsstudiengang studierst, kommen mehrere Fachschaften für dich in Frage: Mathematik, dein zweites Fach und Allgemeinbildende Schulen.
% TODO:  <30-08-17, Bela> wie heißt die Fachschaft für Lehrämtler richtig %
% Zu welcher Fachschaft du gehörst, hast du bei der Immatrikulation entschieden, du kannst es aber jederzeit ändern.
% In erster Linie ist es bei den Hochschulwahlen relevant, denn dort kannst du nur bei deiner offiziellen
% eigenen Fachschaft wählen und gewählt werden.
% TODO:  <30-08-17, Bela> wie funktioniert das? Was ist der richtige Link? %

\subsection{Hochschulgruppen}
\label{sub:hochschulgruppen}
Neben den hochschulpolitischen Vertretungen (FSR, Studentenrat (StuRa)) gibt es zahlreiche Hochschulgruppen,
die sich für verschiedenes einsetzen. Manche stellen sich in diesem Heft vor, alle findet man auf der Internetseite des StuRa\footnote{\url{stura.tu-dresden.de/hochschulgruppen}}.

\section{Angebote der Universität: Checkliste}
\label{sec:angebote_der_universitat}
Die TU Dresden bietet eine Menge Services an, die sehr praktisch für das Studium und auch drum herum sein können.
Mann muss sie nur kennen. Hier ist eine Liste von Angeboten, die für sich einrichten sollte.
Eine deutlich umfangreichere Informationssammlung zum Studienstart
existiert online\footnote{\url{tu-dresden.de/studienstart}} von der Uni.
\newcommand{\clitem}[1]{\item[$\square$ #1]}
% \newcommand{footnoteref}[1]{\footnotemark[\ref{#1}]}
\begin{description}
  \clitem{ZIH-Login aktivieren} Das Zentrum für Informationsdienste und Hochleistungsrechnen (\keyword{ZIH})
    bietet alle IT-Dienste für die TUD an. Man braucht daher einen \keyword{ZIH-Login}, den bei der
    Immatrikulation bekommen hat, aber noch aktiviert werden muss.
    Dieser wird häufig als \keyword{s-Nummer} bezeichnet, da es eine Nummer mit vorangestelltem „s“ ist.
    Informationen dazu und zu allen (folgenden) IT-Diensten finden sich ausführlich auf den Erstsemesterseiten des ZIH \footnote{\url{tu-dresden.de/zih/dienste/service-desk/ese}\label{zihurl}}.
    \clitem{Uni-e-Mail-Adresse} Du hast eine e-Mail-Adresse\footref{zihurl} der Form \url{vorname.nachname@mailbox.tu-dresden.de}.
    Zum Einen ist es praktisch für Kommunikation in Uniangelegenheiten dafür eine e-Mail-Adresse zu haben.
    Zum Anderen bist du von der Uni verpflichtet, diese regelmäßig zu lesen.
    Beispielsweise kommt darüber rechtzeitig die Erinnerung, sich für das kommende Semester zurückzumelden.
    Du kannst die eingehenden e-Mails auf deine private e-Mail-Adresse umleiten oder sie wie andere Adressen
    online oder über einen e-Mail-Client wie Thunderbird abrufen (unsere Empfehlung).
  \clitem{eduroam} Auf dem gesamten Unigelände und in zahlreichen Unis auf der ganzen Welt steht dir
    das WLAN \keyword{eduroam}\footref{zihurl} mit Internetzugang zur Verfügung.
    \clitem{νsletter} Melde dich doch für den Newsletter des μFSR an\footnote{myfsr.de/minitrue/auth/page/email/register\#}.
    % TODO:  <30-08-17, Bela> Kurzlink für νsletter %
    Wir spammen dich nicht voll, sondern
    filtern aus den zahlreichen Angeboten für Studierende die für dich (hoffentlich) spannensten heraus.
    Das ist einer der sinnvollen Verwendungszwecke für die Uni-e-Mail-Adresse.
  \clitem{SLUB-Karte} Die Sächsische Landes- und Universitätsbibliothek (SLUB) hat ihre eigene Benutzerkarte,
    die man sich ausstellen lässt. Du füllst erst das Anmeldeformular aus und holst dann die SLUB-Karte ab. Dabei an Perso und Immatrikulationsbescheinigung denken. Infos auf \url{slub-dresden.de/service/nutzer-der-slub-werden}.
  \clitem{Mensa-Karte} Du kannst theoretisch in der Mensa mit Bargeld bezahlen,
    aber das ist unpraktisch und hält den eh vollen Betrieb auf und
    du müsstest dich jedes Mal als Student ausweisen.
    Deshalb gibt es die Mensa-Karte,
    die du in jeder Mensa für 5€ Kaution bekommst.
    Um Studentenpreise zu zahlen musst du die EMeal-Bescheinigung von deinem Immatrikulationsbogen
    einmal am Anfang jedes Semesters vorzeigen.
    Studentenausweis oder Immatrikulationsbescheinigung reichen nicht.
  \clitem{OFP} Die Orientierungsplattform Forschung und Praxis (OFP) bietet regelmäßig spannende Veranstaltungen
    speziell für uns Mathematikstudierende an.
    Diese kann man leicht übersehen und verpassen.
    Daher ist es eine gute Idee sich für den Newsletter anzumelden.
    Das geht indem man sich bei OPAL anmeldet und zum passenden auf den
    OFP-Seite\footnote{\url{tu-dresden.de/tu-dresden/profil/exzellenz/zukunftskonzept/tud-structures/zill/orientierungsplattform-forschung-praxis/mathematik}}
    verlinkten Kurs\footnote{\url{https://bildungsportal.sachsen.de/opal/auth/RepositoryEntry/12520161286/CourseNode/94473775961368}} geht.
    
    Ja, OPAL ist etwas umständlich zu bedienen.
    \clitem{Sportkurse} Das Universitätssportzentrum (\keyword{USZ})\footnote{\url{tu-dresden.de/usz/zentrale-infos/hinweise-zur-buchung}} bietet ein riesiges Angebot\footnote{\url{www.usz.tu-dresden.de}} an
    Sportkursen an, die für Studierende auch sehr günstig sind.
    Sehr viele dieser Kurse sind sehr beliebt, weshalb man sehr schnell bei der Anmeldung sein muss.
    Sehr schnell bedeutet, dass innerhalb von unter 5 Minuten alle Plätze vergeben sein können.
    % TODO:  <30-08-17, Bela> wie heißt die Schnellanmeldefunktion? Die ist als Kurs registriert. %
    Der Anmeldebeginn steht bei dem jeweiligen Kurs dabei. Die ersten beginnen am 10.10.2017 um 17 Uhr.
\end{description}
Die folgende Punkte muss man nicht unbedingt direkt am Anfang des Semesters beachten, sollte man
sich aber bald anschauen.
\begin{description}
  \item[Nebenfach] Zum Mathematikstudium gehört ein Nebenfach.
    Für dieses gibt es mehrere Vorschläge, die in der Anlage 2 der Prüfungsordnung
    \footnote{\url{www.verw.tu-dresden.de/AmtBek/PDF-Dateien/2016-03/01soBA26.02.2016.pdf\#page=64}}
    nachzulesen sind.
    Du kannst aber auch andere Module wählen, solange sie in der Summe
    die gleiche Anzahl an Leistungspunkten haben.
    Bevor du ein nicht vorgeschlagenes Modul belegst wende dich an
    das Prüfungsamt (Frau Schreiter)\footnote{\url{tu-dresden.de/mn/math/studium/pruefungsaemter}}
    und kläre ob es angerechnet werden kann und auf welche Art
    du dich zur Prüfung anmelden kannst.

    Für dein Nebenfach entscheidest du dich mit deiner ersten
    Prüfungsanmeldung.
  \item[ownCloud] Unter anderem bietet das ZIH dir 2GB Online-Speicher\footref{zihurl} an.
    Dieser läuft unter dem Name CloudStore (beziehungsweise dem Namen der verwendeten Software ownCloud).
    Dieser ist gegenüber kommerziellen Anbietern zu bevorzugen,
    da hierbei die Daten auf Uni-Servern bleiben. Man kann einzelne Ordner zum Austausch freigeben.
  \item[OPAL] Wofür OPAL steht ist nicht so einfach zu ermitteln. Es ist eine Internetseite, die hauptsächlich
    zum Einschreiben in Übungen und Hochladen von Übungszetteln genutzt wird. Theoretisch geht noch viel mehr
    wie Online-Tests und -Übungen.
    Anmelden kannst du dich mit den ZIH-Login-Daten (s-Nummer).
  \item[Semesterticket] Mit dem Semesterticket kannst du den ÖPNV nutzen. Welche Busse, Bahnen und Fahrräder dazugehören und wann du wo ein Fahrrad mitnehmen kannst,
    steht ausführlich auf der StuRa-Seite.\footnote{\url{stura.tu-dresden.de/semesterticket}}.
  \item[Lehrveranstaltungskatalog] Der Lehrveranstaltungskatalog\footnote{\url{www.math.tu-dresden.de/math/lvk/lv-abfrage-wi17.htm}} enthält Orte und Zeiten aller angebotenen
    Vorlesungen, Übungen, Seminare, etc. Jede Fakultät hat einen eigenen.
  \item[Studienfachberatung] Die Studienfachberatung berät zu
    inhaltlichen (Welche Module sind für mich sinnvoll, etc.?)
    und organisatorischen Fragen.
    In einigen solcher Fälle kann der μFSR auch weiterhelfen.

    Für den Bachelor Mathematik bietet diese Dr.\ Hans-Peter Scheffler
    \footnote{\url{tu-dresden.de/mn/math/studium/beratung/bachelor_mathematik}}, für das Lehramt Mathematik Dr.\ Karin Weigel\footnote{\url{tu-dresden.de/mn/math/studium/beratung/bachelor_mathematik}}.
  \item[Zentrale Studienberatung] Die zentrale Studienberatung\footnote{\url{tu-dresden.de/studium/im-studium/beratung-und-service/zentrale-studienberatung}} ist der erste Ansprechpartner wenn über einen Studiengangswechsel nachgedacht wird. Daneben bietet sie aber auch weitere Angebote,
    die man sich anschauen sollte.
    Nichtsdestotrotz haben die meisten innerhalb des Mathematikstudiums keinen Kontakt dahin.
  \item[Career-Service] Der Career-Service\footnote{\url{tu-dresden.de/studium/im-studium/beratung-und-service/career-service}} bietet in erster Linie Unterstützung beim Berufseinstieg.
    Aber er bietet auch zahlreiche Workshops\footnote{\url{tu-dresden.de/karriere/berufseinstieg/ressourcen/dateien/flyer/2017/semesterprogramm-des-career-service-winter-2017-18\#page=2}} an, die bereits im Studium
    sehr hilfreich sein können wie bspw.\ zu Prüfungsbewältigung und
    wissenschaftlichem Arbeiten und Schreiben.
  \item[Schreibzentrum] Das Schreibzentrum bietet Unterstützung beim Schreiben von wissenschaftlichen Arbeiten, siehe auch deren Seite.
  \item[Studiendokumente] Um nicht von bisher unbekannten Regelungen
    überrascht zu werden, empfiehlt es sich, die Studiendokumente
    \footnote{Bachelor: \url{tu-dresden.de/mn/studium/studiendokumente-formulare/mathematik-bachelor}\\
    Lehramt: \url{tu-dresden.de/zlsb/lehramtsstudium/studiendokumente}}
    (Studien- und Prüfungsordnung) zur Verfügung stehen zu haben und
    gelesen zu haben.
  
\end{description}
Wenn die hier gegebenen Informationen mehr auf den Bachelor
Mathematik zugeschnitten sind als auf Lehramt Mathematik,
dann tut uns das Leid. Es liegt daran, dass
die meisten μFSR-ler kein Lehramt studieren und sich deshalb
nicht so gut auskennen. Du kannst das ändern, indem du den
μFSR als gewähltes oder assoziiertes Mitglied unterstützt!
\end{document}
