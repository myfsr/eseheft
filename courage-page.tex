\begin{center}
  \includegraphics[width=\textwidth]{./courage-logo-bw.png}
\end{center}
\section*{Veranstaltungsreihe im Wintersemester 2017/18}
\label{sec:veranstaltungsreihe_courage -- wissen, sehen, handeln!_}
Veranstaltungsreihe für alle Angehörigen der Technischen Universität Dresden im Wintersemester 2017/18

Zwischen Oktober und Dezember 2017 findet das Projekt \enquote{Courage -- wissen, sehen, handeln!} der TU Dresden statt.
In Workshops, Vorträgen und kulturellen Veranstaltungen wird der Frage nachgegangen, wie man Rassismus, Menschenfeindlichkeit und Diskriminierung begegnen kann. Dabei wird auf Möglichkeiten aufmerksam gemacht, Ausgrenzungsphänomene zu erkennen, abzubauen und zu verhindern. 

\keyword{Workshops:} Was kann ich tun, wenn ich Opfer oder Zeuge*in eines rassistischen Übergriffs werde? Was darf ich tun? Wie argumentiere ich schlagfertig gegen rechte Parolen? Wie gehe ich als Betroffene*r mit Rassismus um? Die Workshops geben Antworten und praktische Tipps.

\keyword{Vorträge:} Was ist Rassismus? In welchen Kontexten tritt er auf? Was kann eine Universität gegen Rassismus tun? Die Vorträge von namhaften Expert*innen helfen Rassismus zu verstehen und zu erkennen.

\keyword{Kulturprogramm:} Zahlreiche Veranstaltungen in Kooperation mit Dresdner Kultur- und Bildungsinstitutionen laden zu einem Perspektivwechsel ein.

Das Projekt richtet sich an alle Mitglieder der TU Dresden und soll Studierende, Dozierende und Verwaltungspersonal der TU Dresden für den Umgang mit Diskriminierung, Ausgrenzung und Rassismus sensibilisieren und empowern. Weitere Informationen findet ihr unter \url{http://tu-dresden.de/courage}.
